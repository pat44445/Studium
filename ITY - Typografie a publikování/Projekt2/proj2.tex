\documentclass[twocolumn,a4paper,11pt]{article}
\usepackage[utf8]{inputenc}
\usepackage[left=1.5cm,text={18cm, 25cm},top=2.5cm]{geometry}
\usepackage{amsmath}
\usepackage[czech]{babel}
\usepackage[IL2]{fontenc}
\usepackage{times}
\usepackage{amssymb}
\usepackage{amsthm}


\usepackage{hyperref}
\newtheorem{definice}{Definice}
\newtheorem{veta}{Věta}

\begin{document}

\begin{titlepage}

\begin{center}
\renewcommand{\baselinestretch}{1.5}\normalsize
{\Huge {FAKULTA INFORMAČNÍCH TECHNOLOGIÍ \\ VYSOKÉ UČENÍ TECHNICKÉ V BRNĚ}}\\
\vspace{\stretch{0.382}}
{\LARGE Typografie a publikování – 2. projekt \\ Sazba dokumentů a matematických výrazů}
\\
\vspace{\stretch{0.618}}
\end{center}

{\LARGE 2019 \hfill Tomáš Dvořáček (xdvora3d)}


\end{titlepage}


\section*{Úvod}
V této úloze si vyzkoušíme sazbu titulní strany, matematických vzorců, prostředí a dalších textových struktur obvyklých pro technicky zaměřené texty (například rovnice (1) nebo Definice 1 na straně 1). Pro odkazovaní na vzorce a struktury zásadně používáme příkaz \verb|\label| a \verb|\ref| případně \verb|\pageref| pokud se chceme odkázat na stranu výskytu.
\par Na titulní straně je využito sázení nadpisu podle optického středu s využitím zlatého řezu. Tento postup byl probírán na přednášce. Dále je použito odřádkování se zadanou relativní velikostí 0.4 em a 0.3 em.

\section{Matematický text}
Nejprve se podíváme na sázení matematických symbolů a výrazů v plynulém textu včetně sazby definic a vět s využitím balíku \verb|amsthm|. Rovněž použijeme poznámku \verb|\footnote| pod čarou s použitím příkazu. Někdy je vhodné použít konstrukci \verb|\mbox{}|, která říká, že text nemá být zalomen.

\begin{definice}
Zásobníkový automat (ZA) je definován jako sedmice tvaru A = (\(Q, \Sigma, \Gamma, \delta, q_0, Z_0\), F), kde:
\end{definice}

\begin{itemize}
    \item Q \emph{je konečná} množina vnitřních (řídicích) stavů,
    \item  
     \(\Sigma\) \emph{je konečná}  vstupní abeceda,
    \item
    \(\Gamma\) \emph{je konečná}  zásobníková abeceda,
    \item
    \(\delta\) \emph{je} přechodová funkce \(Q \times (\Sigma\cup\verb|{|\epsilon\verb|}|) \times \Gamma \rightarrow 2^{Q\times\Gamma^*}\),
    \item
    \(q_0\in\) Q je počáteční stav, \(Z_0\in\Gamma\) je startovací symbol zásobníku a \(F\subseteq Q\) \emph{je množina} koncových stavů.
\end{itemize}

\par Nechť \(P=(Q,\Sigma,\Gamma,\delta,q_0,Z_0,F)\) je zásobníkový automat. \emph{Konfigurací} nazveme trojici\((q,w,\alpha)\in Q\times\Sigma^*\times\Gamma^*\), kde \emph{q} je aktuální stav vnitřního řízení, \emph{w} je dosud nezpracovaná část vstupního řetězce a \( \alpha=Z_{i_1}Z_{i_2}\)\dots\(Z_{i_k}\) je obsah zásobníku\footnote{$Z_{i_1}$ je vrchol zásobníku}.

\subsection{Podsekce obsahující větu a odkaz}
\begin{definice}
\emph{Řetězec} w \emph{nad abecedou} \(\Sigma\) je přijat ZA \emph{A} \emph{jestliže}\((q_0,w,Z_0)\) $\vdash_{A}^{*}$ \((q_F,\epsilon,\gamma)\) pro nějaké \(\gamma \in \Sigma^*\) a \(q_F\in F\). Množinu L(A) =\(\verb|{|w|w\) je přijat ZA \emph{A}\verb|}| \(\subseteq\Gamma^*\) nazýváme \emph{jazyk přijímaný TS} M.


\end{definice}

\par Nyní si vyzkoušíme sazbu vět a důkazů opět s použitím balíku \texttt{amsthm}

\begin{veta}
Třída jazyků, které jsou přijímány ZA, odpovídá \emph{bezkontextovým jazykům.}
\end{veta}
\begin{flushleft}
\emph{Důkaz.} V důkaze vyjdeme z Definice 1 a 2. \hfill$\square$
\end{flushleft}
\section{Rovnice a odkazy}
Složitější matematické formulace sázíme mimo plynulý text. Lze umístit několik výrazů na jeden řádek, ale pak je třeba tyto vhodně oddělit, například příkazem \verb|\quad|.
\break

$\sqrt[i]{x_i^3}$  kde $x_i$ je $i$-té sudé číslo splňující  $x_i^{2-x_i^{i^2}} \leq x_i^{y_i^{3}}$
\break
\par V rovnici (1) jsou využity tři typy závorek s různou explicitně definovanou velikostí.


\begin{equation}
x = \biggl[\Big\{ [a+b]*c\Big\}^d\ominus1\biggr]^{1/2}
\end{equation}


$$y = \lim_{ x\to\infty} \frac{\frac{1}{\log_{10} x}}{\sin^2x+\cos^2x}$$



\par V této větě vidíme, jak vypadá implicitní vysázení limity $\lim_{ x\to\infty} f(n)$ v normálním odstavci textu. Podobně je to i s dalšími symboly jako $\prod_{i=1}^n 2^i$ či $\bigcap_{A\epsilon\beta} A$. V případě vzorců $\lim\limits_{ x\to\infty} f(n)$ a $\prod\limits_{i=1}^n 2^i$ jsme si vynutili méně úspornou sazbu příkazem \verb|\limits|.

\begin{equation}
\int_{b}^{a} g(x) dx\ =\ -\int\limits_{a}^{b} f(x) dx
\end{equation}
\begin{equation}
 \overline{\overline{A \land B}} \,\Leftrightarrow \,\overline{\overline{A} \land \overline{B}} 
\end{equation}
\section{Matice}
Pro sázení matic se velmi často používá prostředí \verb|array| a závorky \verb|(\left,\right)|.
$$
\begin{bmatrix}
& \widehat{\beta+\gamma} & \hat{\pi}  \\
\overrightarrow{a}  & \overleftrightarrow{AC}           
\end{bmatrix} 
=  1 \iff \mathbb{Q}  = \mathbf{R}$$
$$
\mathbf{A} =
\begin{vmatrix}
    a_{11} & a_{12}  & \dots  & a_{1n} \\
    a_{21} & a_{22}  & \dots  & a_{2n} \\
    \vdots & \vdots & \ddots & \vdots \\
    a_{m1} & a_{m2}  & \dots  & a_{mn}
\end{vmatrix}
=\
\begin{matrix}
t & u \\
v & w
\end{matrix}
\
=tw - uv
\break
$$
Prostředí \verb|array| lze úspěšně využít i jinde.
$$
\binom{n}{k} = \left\{ \begin{array}{ll}
         0 & \textrm{pro}\ k < 0\ \textrm{nebo}\ k > n\\
        \frac{n!}{k!(n-k)!} & \textrm{pro}\ 0 \leq k \leq n.\end{array} \right.$$












\end{document}
